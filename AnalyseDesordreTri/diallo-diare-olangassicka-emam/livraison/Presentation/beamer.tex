\documentclass{beamer}
\usepackage[utf8]{inputenc}
\usetheme{Warsaw}

\title{Analyse des algorithmes de tris}
\author{DIARE Youssouf \\ \and DIALLO Boubacar Sadio\\ \and Olangassicka Franck Loick\\ \and EMAM Mohamed El Mamy} 
\institute{Université de Caen Normandie\\
L3 Info\\ Groupe 2A}

\begin{document}

\begin{frame}
\titlepage
\end{frame}


\begin{frame}
\tableofcontents
\end{frame}


\section{Présentation du projet}

\subsection{Problématique}

\begin{frame}
%
\begin{block}{Problematique }
	Je suis la problematique .
\end{block}

\end{frame}



\subsection{Objectifs}

\begin{frame}
%
\begin{exampleblock}{Les objectifs }

	ici il y aura la listes des objectifs 
	\end{exampleblock}

\end{frame}




\section{Décomposition des taches }

\subsection{Générateur de donnée (désordre)}

\begin{frame}

	cette partie parle du generateur de desordre 

\end{frame}




\subsection{Implémentation des algorithmes }

\begin{frame}

	\begin{block}{Implementation des algorithmes }

	cette partie parlera des algos que l'on a implementer et si on a utiliser un design paterne pour le faire 

	\end{block}

\end{frame}




\subsection{visualisation }

\begin{frame}
	\begin{block}{visualisation des algorithme}

	cette partie parle de la visualisation des algos 
	

	\end{block}
\end{frame}




\subsection{Le Héros}

\begin{frame}

	\begin{block}{Initialisation du Héros}

	\begin{figure}[h]
	\includegraphics[scale=0.6]{images/CodeHerosInit.png}
	\caption{Extrait code init() class Heros}
	\end{figure}

	\end{block}


\end{frame}
\begin{frame}

	\begin{block}{Fonctions du Héros}

	\begin{figure}[h]
	\includegraphics[scale=0.4]{images/CodeHerosVieHeros.png}
	\caption{Extrait code la fonction VieHeros de la class Heros}
	\end{figure}

	\end{block}


\end{frame}

\begin{frame}

	\begin{block}{Fonctions du Héros}

	\begin{figure}[h]
	\includegraphics[scale=0.6]{images/CodeHerosMettreDeg.png}
	\end{figure}

	\begin{figure}[h]
	\includegraphics[scale=0.6]{images/CodeHerosAffich.png}
	\end{figure}

	\begin{figure}[h]
	\includegraphics[scale=0.5]{images/CodeHerosVict.png}
	\caption{Extrait code fonctions de la class Heros}
	\end{figure}

	\end{block}

\end{frame}




\subsubsection{L'Ennemi}

\begin{frame}

	\begin{block}{Initialisation de l'Ennemi}

	\begin{figure}[h]
	\includegraphics[scale=0.6]{images/CodeEnnInit.png}
	\caption{Extrait code init() de la class Ennemi}
	\end{figure}

	\end{block}

\end{frame}




\section{Début de l’expérimentation }

\begin{frame}
\begin{block}{Experimentation }
	blabla sur l'expe
\end{block}
\end{frame}


\section{conclusion }

\begin{frame}

\begin{block}{.}
\textbf{conclusion}
\end{block}

\end{frame}
\end{document}
